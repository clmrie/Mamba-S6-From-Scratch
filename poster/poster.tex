\documentclass[final]{beamer}
\usepackage[size=custom,width=90,height=120,scale=1.21]{beamerposter}
\usepackage{graphicx}
\usepackage{booktabs}
\usepackage{tabularx}
\usepackage{amsmath}
\usepackage{algorithm}
\usepackage{algpseudocode}
\usepackage{float} % for [H] placement in algorithm
\usepackage{ragged2e}
\usepackage[dvipsnames]{xcolor} % 为了更好的颜色支持(teal 等命名颜色)

\usetheme{default}
\setbeamercolor{block title}{fg=white,bg=teal} 
\setbeamercolor{block body}{fg=black,bg=white}
\setbeamerfont{block title}{size=\huge}
\setbeamerfont{block body}{size=\Large}
\setlength{\parskip}{1.1em}

\setbeamertemplate{navigation symbols}{}
\setbeamertemplate{caption}[numbered]
\setlength{\columnsep}{0.5cm}


\newcommand{\placeholderbox}[2]{%
  \fbox{\parbox[c][#2][c]{#1}{\centering \textcolor{gray}{Image Placeholder}}}%
}

\title{Selective State Spaces: Bridging the Gap Between RNNs and Transformers}
\author{Clement Marie, Ying JIN}
\institute{Course: Deep Learning Project}

\begin{document}
\begin{frame}[t]
\centering
\vspace*{0.8cm}
{\Huge \bfseries\color{teal} Selective State Spaces}\par
\vspace{0.1cm}
{\huge \bfseries\color{teal} Bridging the Gap Between RNNs and Transformers}\par
\vspace{0.35cm}
{\Large \color{gray!20!black} Clement Marie \quad\textbullet\quad Ying JIN}\par
\vspace{0.12cm}
{\large \color{gray!40!black} Course: Deep Learning Project}\par
\vspace{0.45cm}
{\color{teal}\rule{0.84\linewidth}{1.8pt}}\vspace{0.45cm}

\begin{columns}[t,totalwidth=\textwidth]

% ============================================================
% Column 1: Motivation & Core Method (Theory)
% ============================================================
\begin{column}{0.33\textwidth}
  \begin{block}{1. Motivation \& Goal}
    \justifying
    \textbf{The Problem:} While Transformers dominate deep learning, their quadratic complexity ($O(N^2)$) limits scalability. Mamba (SSM) offers linear-time ($O(N)$) inference, but official implementations are opaque (CUDA-heavy) and strictly causal (unidirectional).
    
    \vspace{0.5em}
    \textbf{Our Goal:} 
    \begin{enumerate}
        \item Provide a transparent \textbf{Pure PyTorch} implementation of Selective Scan (S6).
        \item Extend Mamba with \textbf{Bi-directional Inductive Bias} for Vision tasks.
    \end{enumerate}
  \end{block}

  \begin{block}{2. Method: Mamba Architecture}
    \justifying
    Mamba replaces attention with an input-dependent SSM. Key components: \textbf{Learned Discretization $\Delta$}, \textbf{Causal Conv1d}, and \textbf{Gating}.
    \vspace{0.5em}
    
    \begin{figure}[h]
      \centering
      \IfFileExists{../outputs/figures/mamba_arch.jpg}{\includegraphics[width=0.98\linewidth]{../outputs/figures/mamba_arch.jpg}}{\placeholderbox{0.98\linewidth}{5cm}}
      \vspace{0.6cm}
      \IfFileExists{../outputs/figures/selective state space model.jpg}{\includegraphics[width=0.98\linewidth]{../outputs/figures/selective state space model.jpg}}{\placeholderbox{0.98\linewidth}{5cm}}
      \caption{(Top) Gated block with SiLU and residual; (Bottom) Selection mechanism where learned $\Delta$ acts as a gate.}
    \end{figure}

    \textbf{From Continuous to Discrete:}
    Discretizing the continuous ODE $h'(t) = \mathbf{A}h(t) + \mathbf{B}x(t)$ with a learned time-step $\Delta$:
    \vspace{-0.2em}
    \begin{equation*}
      \begin{aligned}
        h_t &= \overline{\mathbf{A}}_t h_{t-1} + \overline{\mathbf{B}}_t x_t, \quad y_t = \mathbf{C}_t h_t \\
        \text{where } \overline{\mathbf{A}}_t &= \exp(\mathbf{\Delta}_t \mathbf{A}), \quad \overline{\mathbf{B}}_t = (\mathbf{\Delta}_t \mathbf{A})^{-1}(\overline{\mathbf{A}}_t - \mathbf{I}) \cdot \Delta_t \mathbf{B}_t
      \end{aligned}
    \end{equation*}
    \vspace{-0.2em}
    {\textit{Innovation:} $\Delta_t$ depends on input $x_t$, allowing the model to selectively "remember".}
  \end{block}


  \begin{block}{3. Algorithm: Selective Scan (S6)}
    \justifying
    The input-dependent recurrence is computed via a parallel scan.
    \vspace{0.2em}
    {\footnotesize
    \begin{algorithm}[H]
      \caption{SSM + Selection (S6)}
      \begin{algorithmic}[1]
        \Require Input $x: (B, L, D)$
            \Ensure Output $y: (B, L, D)$
            
            \State $A: (D, N) \gets \text{Parameter}$ \Comment{Fixed structured matrix}
            
            \Statex \textcolor{blue}{\textit{// 1. Selection: Projects input to parameters}}
            \State $B: (B, L, N) \gets \text{Linear}_B(x)$
            \State $C: (B, L, N) \gets \text{Linear}_C(x)$
            \State $\Delta: (B, L, D) \gets \text{Softplus}(\text{Parameter} + \text{Linear}_\Delta(x))$
            
            \Statex \textcolor{blue}{\textit{// 2. Discretization}}
            \State $\overline{A}, \overline{B} \gets \text{discretize}(\Delta, A, B)$
            
            \Statex \textcolor{blue}{\textit{// 3. Selective Scan (The parallel recurrence)}}
            \State $y \gets \text{SSM}(\overline{A}, \overline{B}, C)(x)$
        \State \Return $y$
      \end{algorithmic}
    \end{algorithm}
    }
  \end{block}
\end{column}

% ============================================================
% Column 2: Implementation & Vision Extension (Engineering)
% ============================================================
\begin{column}{0.33\textwidth}

  \begin{block}{4. Implementation Details}
    \begin{itemize}
      \item \textbf{Framework:} Pure PyTorch (No custom CUDA kernels).
      \item \textbf{Conv1d:} Kernel size 4, helps with local context.
      \item \textbf{Normalization:} RMSNorm for stability.
      \item \textbf{Optimization:} Weight tying between embeddings and output head.
    \end{itemize}

  The code can be found at : https://github.com/clmrie/Mamba-S6-From-Scratch.
  \end{block}

  \vspace{1.5cm}

  \begin{block}{5. Vision Extension: Bi-Directional}
    \justifying
    \textbf{Why Bi-directional?} Standard Mamba lose context from "future" pixels, which is suboptimal for images. Bi-directional scanning captures global spatial context. We implement a \textbf{Bi-Directional Vision Mamba}:
    \begin{enumerate}
        \item Flatten image into patch tokens.
        \item Scan Forward ($ \rightarrow $) and Backward ($ \leftarrow $).
        \item Fuse branches (Average/Concat).
    \end{enumerate}
    \vspace{0.5em}
     \vspace{0.5em}
    \begin{figure}
      \centering
      \IfFileExists{../outputs/figures/ablation_study.png}{\includegraphics[width=0.95\linewidth]{../outputs/figures/ablation_study.png}}{\placeholderbox{0.95\linewidth}{4cm}}
      \caption{Causal vs. Bi-Directional Performance.}
    \end{figure}
  \end{block}
  
  \vspace{4.5cm}

  \begin{block}{6. Experimental Setup}
    \begin{itemize}
      \item MNIST: Vision Mamba, causal ablation, vanilla RNN baseline.
      \item CIFAR-10: patch-based classification.
      \item TinyShakespeare: Character-level LM.
    \end{itemize}

    \vspace{0.5em}

    \begin{table}

      \centering

      {\small

        \begin{tabular}{p{0.30\linewidth}p{0.62\linewidth}}

          \toprule

          \textbf{Dataset} & \textbf{Key training settings} \\

          \midrule

          MNIST (Vision) & d\_model=64, layers=2, batch=64, lr=1e-3, epochs=5 \\

          MNIST (Causal) & d\_model=64, layers=2, batch=64, lr=3e-4, epochs=5 \\

          MNIST (RNN) & input=16, hidden=64, layers=2, batch=64, lr=1e-3, epochs=5 \\

          CIFAR-10 & d\_model=128, layers=4, batch=64, lr=1e-3, epochs=5 \\

          TinyShakespeare & d\_model=128, layers=4, block=128, batch=32, lr=1e-3, epochs=10 \\

          \bottomrule

        \end{tabular}
      }

      \caption{Training settings summary}

    \end{table}
  \end{block}
\end{column}

% ============================================================
% Column 3: Results & Conclusion (Evidence)
% ============================================================
\begin{column}{0.33\textwidth}
  \begin{block}{7. Results: Mamba vs RNN}
    \justifying
    Vision Mamba converges significantly faster than vanilla RNNs and achieves higher final accuracy.
    \vspace{0.5em}
    \begin{figure}
      \centering
      \IfFileExists{../outputs/figures/mamba_vs_rnn.png}{\includegraphics[width=0.95\linewidth]{../outputs/figures/mamba_vs_rnn.png}}{\placeholderbox{0.95\linewidth}{5cm}}
      \caption{Training Loss Comparison on MNIST.}
    \end{figure}

    \begin{table}
      \centering
      {\small
        \begin{tabular}{lc}
          \toprule
          \textbf{Model Architecture} & \textbf{Acc. (\%)} \\
          \midrule
          Vanilla RNN & 94.40 \\
          Causal Vision Mamba & 97.00 \\
          \textbf{Bi-Directional Vision Mamba (Ours)} & \textbf{97.93} \\
          \bottomrule
        \end{tabular}
      }
      \caption{Classification Accuracy.}
    \end{table}
  \end{block}

  \begin{block}{7. Results: Visulization of delta selection}
    \justifying
    We analyze the magnitude of the learned time-step $\Delta_t$ on a "Needle in a Haystack" task.
The model exhibits distinct \textbf{activation spikes} (high $\Delta$ values) precisely at the positions of critical information (the "needle"), while maintaining low values for irrelevant noise.
Mathematically, a large $\Delta_t$ increases the update rate of the state $h_t$, effectively \textbf{"opening the gate"} to write information into memory.
    \vspace{0.5em}
    \begin{figure}
      \centering
      \IfFileExists{../outputs/figures/delta_visualization.png}{\includegraphics[width=0.95\linewidth]{../outputs/figures/delta_visualization.png}}{\placeholderbox{0.95\linewidth}{4cm}}
      \caption{Visualization of $\Delta$}
    \end{figure}
  \end{block}

  \begin{block}{8. Conclusion}
    \begin{itemize}
      \item Successfully reimplemented \textbf{Mamba S6 from scratch} without CUDA kernels.
      \item \textbf{Bi-directional scanning} is crucial for non-causal data (images), boosting accuracy by $\sim$1\% over causal Mamba.
      \item Demonstrated linear-time scaling potential on sequence tasks.
    \end{itemize}
  \end{block}

  \begin{block}{References}
    \footnotesize
    [1] Gu \& Dao, \emph{Mamba: Linear-Time Sequence Modeling with Selective State Spaces}, 2023.\\[0.4em]
  \end{block}
\end{column}

\end{columns}
\end{frame}
\end{document}